Ardu\+Viz and \hyperlink{class_app_builder}{App\+Builder} are libraries intended to provide an easy way of vizualizing data and informations from your arduino with the help of your P\+C or Android phone (coming soon!). Additionally, you can interact with your arduino via G\+U\+I -\/ inlcuding buttons, text inputs and labels -\/ and send it commands or even control it.

At its core, \hyperlink{class_app_builder}{App\+Builder} and Ardu\+Viz are designed to be as versatile as possible. It runs on any platform (Windows, Mac, Linux) and supports most arduinos. Furthermore, you're theoretically able to customize both libraries to your liking, e.\+g. adding, altering or removing elements at whim.

This documentation is centered around the use of the serial connection between arduino and client.

  

The layout of your biult app will always look the same, regardless of your operating system. (Tested on Windows 7, Mac O\+S 10.\+10, Ubuntu 14.\+04)

Also, the handling and usage will not differ between these platforms. However, you have to have the same preconidtions (listed below) installed on every system. The tutorial will explain what you need and where to get it.

\subsection*{Features}


\begin{DoxyItemize}
\item G\+U\+I powered by the \href{http://kivy.org}{\tt Kivy} library (with huge potential of extensions, e.\+g.\+: diagram,progressbar ,pictures,...)
\item Two connection modes\+: connection and auto reconnection
\item Connect multiple arduinos to one P\+C
\item Multi-\/platform (Linux,Windows,Mac O\+S)
\item Well documented
\item Easily extensible
\item Fast start tutorial
\end{DoxyItemize}

\subsection*{What you need to start}


\begin{DoxyItemize}
\item Python 2.\+7
\item \href{http://kivy.org}{\tt Kivy}
\item \href{http://pyserial.sourceforge.net}{\tt py\+Serial}
\end{DoxyItemize}

We have also made a detailed installtion guide \href{md_documentation_appb_intro.html}{\tt here}.

\subsubsection*{\hyperlink{class_app_builder}{App\+Builder}\+:}

The functionality of the \hyperlink{class_app_builder}{App\+Builder} (arduino library) can be found \href{md_documentation_installation.html}{\tt here}.

\subsubsection*{Ardu\+Viz\+:}

The functionality of the Ardu\+Viz application (client) can be found \href{md_documentation_arduviz_intro.html}{\tt here}. 