\hyperlink{class_app_builder}{App\+Builder} is a library that's designed to communicate with a compatible counterpart, e.\+g. a computer, an android phone or even another arduino handset.

Currently supported are the following functions\+:
\begin{DoxyItemize}
\item Connection detection (disconnction detection isn't currently supported due to restriction from arduino's serial library)
\item Handshake to ensure that you're connected to a compatible and trustworhty counterpart
\item Building layouts and sending them (You can nested layouts etc. The view, though, depends on your counterpart)
\item Setting, receiving and executing callback for both buttons and textinputs (the latter also receives the string that as inside the textinput)
\item Setting text
\end{DoxyItemize}

There are several features still to be implemented like creating diagrams and sending data continously. The latter can be emulated by sending text.

\section*{Usage}

The library was designed to be as straighforward as possible, so all names should be self-\/explanatory and easy to understand. Nonetheless, a tutorial on how to use the library will beneift your understanding.

\subsection*{Getting started}

\subsubsection*{Initializing}

\hyperlink{class_app_builder}{App\+Builder}'s constructor is defined as follows\+:


\begin{DoxyCode}
\hyperlink{class_app_builder}{AppBuilder}(byte num\_com, byte num\_callbacks, String handshake, \textcolor{keywordtype}{int} attempts);
\end{DoxyCode}


The arguments explained\+:
\begin{DoxyItemize}
\item {\ttfamily num\+\_\+com} is the {\itshape number of components} you want to use
\item {\ttfamily num\+\_\+callbacks} is the {\itshape number of callbacks} you want to use
\item {\ttfamily handshake} is the {\itshape handshake phrase} that's used to connect to a counterpart
\item {\ttfamily attempts} is the {\itshape number of attempts} made at doing a handshake
\end{DoxyItemize}

An exmaple for initializing your \hyperlink{class_app_builder}{App\+Builder} object would look like this\+:


\begin{DoxyCode}
\hyperlink{class_app_builder}{AppBuilder} appb(12, 5, String(\textcolor{stringliteral}{"AVH\(\backslash\)0"}), 2048);
\end{DoxyCode}


You can store {\itshape 12} components that have {\itshape 5} callbacks. The connection phrase would be {\itshape A\+V\+H} (Arduino\+Viz\+Handshake) and the number for attempts {\itshape 2048}.

Next thing you need would be a function that's called when you're connected. That's probably your first encounter with gen\+\_\+callback. It will be explained later on.

\subsubsection*{Building a layout}

\subsection*{Setting properties}

\subsubsection*{Setting the text of somehting}

\subsection*{Setting and receiving callbacks}

\subsubsection*{Setting callbacks}

\subsubsection*{Receiving callbacks and handling them}

\section*{Callbacks explained}

\subsection*{When to expect them}

\section*{General sequence of events}